\documentclass[letter,12pt]{exam}
\usepackage{gensymb}
\begin{document}

\begin{center}
    \fbox{\fbox{\parbox{5.5in}{\centering
    Chapter 8 Test Solutions}}}
\end{center}

\section{Multiple Choice}

\begin{questions}
	\question A. $F_m=Bqv\sin(\theta)$ and $\sin(0)=0$.
	\question D. Substitute the numbers into the formula $F_m=Bqv\sin(\theta)$.
		$$F_m=(500)(175)(\frac{1000}{3600})\sin(0)$$
		$$F_m=2.43\times10^4$$
	\question D. Use the formula $F=IlB\sin(\theta)$
		$$1=I(0.025)(1)$$
		$$I=40$$
	\question A. $B=\mu_0(\frac{NI}{L})$
		$$2.4 \times 10^{-2}=\mu_0(\frac{15(I)}{0.15})$$
		$$I=179$$
	\question D. By $B=\mu_0(\frac{I}{2\pi r})$, we see only A and B are true.
		
\end{questions}

\section{Full Solution}

\subsection{Question 1}

A particle with mass $m$ and charge $e$ is launched out of a device with a velocity of $v$ m/s into deep space. After a while, the particle enters a magnetic field with a strength of B [into page] at an angle of $45 \degree$ above the horizontal.

\begin{parts}

	\part[1] Using the variables given above, what is the $\vec{F}_m$ experienced by the particle?
	$$\vec{F_m} = q(\vec{B} \times \vec{v})$$
	$$|F_m| = Bev\sin(45 \degree)$$
	$$\vec{F_m} = Bev\frac{\sqrt{2}}{2}[\rightarrow]$$

	\part[2] Describe the path the particle follows after a lengthy period of time causes the particle to settle into a determinable path.

		\begin{enumerate}
		\part Top-down view will be circular
		\part Rising spiral
	\end{enumerate}

	One point for each of the above.

	\part[1] After a period of time, the particle settles into a path discussed in the previous part. Determine the radius of this orbit.
	$$F_c=F_m$$
	$$\frac{mv^2}{r}=Bev\frac{\sqrt{2}}{2}$$
	$$r=\frac{\sqrt{2}mv^2}{Bev}$$

	\part[2] If the mass was halved to $\frac{m}{2}$ and the magnetic field was reversed to be [out of page], describe the new path the particle follows.

	\begin{enumerate}
		\part Half radius
		\part Direction of orbit reversed
	\end{enumerate}

	One point for each of the above.

\end{parts}

\end{document}