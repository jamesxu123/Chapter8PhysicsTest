\documentclass[letter,12pt]{exam}
\usepackage{gensymb}
\usepackage{tikzit}

\input{styles/particle.tikzstyles}

\begin{document}

\begin{center}
    \fbox{\fbox{\parbox{5.5in}{\centering
    Answer the questions in the spaces provided. If you run out of room
    for an answer, continue on the back of the page.}}}
\end{center}

\vspace{5mm}

\makebox[\textwidth]{Name:\enspace\hrulefill}

\vspace{5mm}

\makebox[\textwidth]{Course code and period:\enspace\hrulefill}

\section{Multiple Choice}
\begin{questions}

\question A charged particle with mass $m$ passes into a magnetic field parallel to the field. What is the magnetic force the particle experiences?

\begin{oneparchoices}
    \choice 0 N
    \choice $+ \infty$
    \choice $- \infty$
    \choice $m$ N
    \choice 1 N
\end{oneparchoices}

\question A space fighter is flying perpendicularly through an enemy defence magnetic field of $5.00 \times 10^2$ T at a speed of $1.00 \times 10^3$km/h, and is struck by a charged particle lance, charging the fighter with 175C. What is the magnitude of the net force experienced by the fighter?

\begin{oneparchoices}
    \choice 0 N
    \choice $2.43 \times 10^4$ N
    \choice $8.75 \times 10^7$ N
    \choice $2.43 \times 10^2$ N
    \choice 1 N
\end{oneparchoices}

\question What current is needed for a conductor of length 0.025m, perpendicular to (and fully in) a magnetic field with a strength of 0.10 T, to have a force of 1 N?

\begin{oneparchoices}
    \choice 0 A
    \choice 400 A
    \choice 4 A
    \choice 40 A
    \choice 1 N
\end{oneparchoices}

\question A solenoid of 16 turns that is 15cm long creates a magnetic field with a strength of $2.4 \times 10^{-2}$ T. What is the current flowing through this solenoid?

\begin{oneparchoices}
    \choice 179 A
    \choice 1790 A
    \choice 17900 A
    \choice 1.79 A
    \choice None of the above
\end{oneparchoices}

\question For a straight conductor of length $L$ and current $I$ and at a distance $r$, which of the following would cause a doubling in magnetic field strength?

\begin{oneparchoices}
    \choice Halving $r$ to $\frac{1}{2}r$
    \choice Doubling $I$ to $2I$
    \choice Halving $L$ to $\frac{1}{2}L$
    \choice A and B
    \choice None of the above
\end{oneparchoices}

\end{questions}

\section{Full Solution}
\begin{questions}

\question[6] A particle with mass $m$ and charge $e$ is launched out of a device with a velocity of $v$ m/s into deep space. After a while, the particle enters a magnetic field with a strength of B [into page] at an angle of $45 \degree$ above the horizontal. The diagram below shows the scenario in the question.

\ctikzfig{q3}

\begin{parts}

    \part[1] Using the variables given above, what is the $\vec{F}_m$ experienced by the particle?

    \vspace{\stretch{1}}

    % \clearpage

    \part[2] After a period of time, the particle settles into a rising circular orbit. Determine the radius of this orbit.

    \vspace{\stretch{1}}

    \part[1] If the mass was halved to $\frac{m}{2}$ and the magnetic field was reversed to be [out of page], describe the new path the particle follows.

    \vspace{\stretch{1}}


    \part[2] Determine the work done when the particle completes a $\frac{1}{4}$ revolution using the situation in (c).

    \vspace{\stretch{1}}

\end{parts}

\end{questions}

\end{document}